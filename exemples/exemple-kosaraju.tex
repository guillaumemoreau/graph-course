\documentclass[xcolor=dvipsnames]{beamer}

\usepackage{tikz}
\usetikzlibrary{graphs}
\usetikzlibrary{graphdrawing.force}
\usetikzlibrary{arrows}
\usepackage{tikz-network}

\title{Algorithme de Kosaraju}
\author{Guillaume Moreau}

\begin{document}
    \maketitle


    
    \directlua{\detokenize{
        for i=0,24,1 do
            tex.sprint("\\begin{frame}\\begin{center} \\input{genfig/kosaraju-", i, ".tex} \\end{center} \\end{frame}")
        end
    }}

    \begin{frame}{Graphe d'origine}

        \begin{center}
            \input{genfig/kosaraju-0.tex}      
        \end{center}
    
    
    \end{frame}
    

\begin{frame}{Ordre suffixe}

    \begin{center}
        \input{genfig/kosaraju-p1.tex}      
    \end{center}
    ordre suffixe : ['3', '18', '17', '11', '10', '15', '9', '14', '8', '20', '19', '13', '7', '1', '2', '4', '5', '12', '6', '22', '16', '21', '24', '23']


\end{frame}

\begin{frame}{Graphe inverse}

    \begin{center}
        \input{genfig/kosaraju-inverse.tex}      
    \end{center}


\end{frame}

\directlua{\detokenize{
    for i=25,48,1 do
        tex.sprint("\\begin{frame}\\begin{center} \\input{genfig/kosaraju-", i, ".tex} \\end{center} \\end{frame}")
    end
}}

\begin{frame}{Composantes fortement connexes}

    \begin{center}
        \input{genfig/kosaraju-cfc.tex}      
    \end{center}


\end{frame}
\end{document}