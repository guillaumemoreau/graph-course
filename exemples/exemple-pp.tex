\documentclass[xcolor=dvipsnames]{beamer}

\usepackage{tikz}
\usetikzlibrary{graphs}
\usetikzlibrary{graphdrawing.force}
\usetikzlibrary{arrows}
\usepackage{tikz-network}
% tentative pour les algorithmes 
\usepackage{algorithm}
\usepackage{algpseudocode}

% package ams 
\usepackage{amssymb}
\def\nbR{\ensuremath{\mathrm{I\! R}}}

\title{Exemples de parcours en profondeur}
\author{Guillaume Moreau}

\begin{document}
    \maketitle

    \begin{frame}[fragile]
        \frametitle{Parcours en profondeur}
            \begin{itemize}
                \item On utilise un tableau de booléens 
                \begin{itemize}
                    \item \emph{vrai} si le sommet a été vu 
                    \item \emph{faux} sinon 
                \end{itemize}
                \item on construit une fonction récursive \texttt{visite(i)} qui va explorer les sommets non vus à partir de $i$ ($i$ inclus)
            \end{itemize}
            
            % test
            \begin{algorithm}[H]
                \begin{algorithmic}[1]
                    \Function{visite}{$i$ : sommet}
                    \State vu[i] \gets true
                    \For{$j \in Adj[i]$}
                        \If{$vu[j] = false$}
                            \State visite(j)
                        \EndIf
                    \EndFor
                    \EndFunction
                 \end{algorithmic}
                \caption{Parcours en profondeur à partir d'un sommet}
                \label{alg:prof:visite}
                \end{algorithm}
        \end{frame}
        
        
        
        % ajouter la version avec les dates de visite ? pas sûr de l'intérêt ici...
        
        \begin{frame}{Parcours en profondeur}
            \begin{itemize}
                \item Attention : l'algorithme n'est pas correct pour parcourir tout le graphe
                \item De fait, il ne fonctionne que si le graphe est connexe 
                \pause 
                \item On peut ajouter une boucle sur tous les sommets 
                \pause 
                \begin{algorithm}[H]
                    \begin{algorithmic}[1]
                        \State vu \gets [$false$,...,$false$]
                        \For{$i \in S$}
                            \If{$vu[i] = false$}
                                \State visite(i)
                            \EndIf
                        \EndFor
                     \end{algorithmic}
                    \caption{Parcours en profondeur}
                    \label{alg:prof:alg}
                    \end{algorithm}
                \pause 
                \item convention implicite : on parcourt les sommets et les adjacents en ordre croissant
            \end{itemize}
        \end{frame}
        
        
         
        \begin{frame}{Efficacité de l'algorithme}
            \begin{itemize}
                \item On marque $vu[i]$ une seule fois 
                \begin{itemize}
                    \item Donc \texttt{visite(i)} termine toujours
                    \item \texttt{visite(i)} n'est appelée qu'une seule fois par sommet
                \end{itemize}
                \item Le temps de calcul est dominé par les itérations de boucles
                \begin{itemize}
                    \item La boucle principale s'exécute $n$ fois 
                    \item La boucle secondaire s'exécute globalement $\sum_i | Adj[i] | = m$
                \end{itemize}
                \item La complexité pire cas est en $n+m$, donc linéaire
            \end{itemize}
        \end{frame}
        

    \directlua{\detokenize{
        for i=0,24,1 do
            tex.sprint("\\begin{frame}\\begin{center} \\input{../genfig/pp-", i, ".tex} \\end{center} \\end{frame}")
        end
    }}

        % \directlua{\detokenize{
        %     for i=0,30,1 do
        %         tex.sprint("\\begin{frame}\\begin{center} \\includegraphics[height=\\textheight]{genfig/rg-", i, ".pdf} \\end{center} \\end{frame}")
        %     end
        % }}
\end{document}