% Définitions pour commencer

\begin{frame}{Graphe}
    \label{def:graphe}
    \begin{definition}
        Un \emph{graphe orienté} est un ensemble de \emph{sommets} connectés par des \emph{arcs}. Formellement, $G=(S,A)$ avec
        \begin{itemize}
            \item $S$ un ensemble de sommets 
            \item $A$ une relation binaire sur $S$ (donc une partie de $S \times S$)
        \end{itemize}
    \end{definition}

    \begin{example}
        \begin{tikzpicture}
            \node[lettre] (1) at (0,0)  {1};
            \node[lettre] (2) at (0,-1) {2};
            \node[lettre] (3) at (1,-1) {3};
            \node[lettre] (4) at (1,0) {4};
            \node[lettre] (5) at (2,0) {5};
            \node[lettre] (6) at (2,-1) {6}; 
            \node (s) at (6,0) {6 sommets, 7 arcs};
            \draw[edge] (1) -> (4);
            \draw[edge] (1) -> (2) -> (4);
            \draw[edge] (2) -> (3);
            \draw[edge] (6) -> (5);
            \draw[edge] (4.260) -> (3.100);
            \draw[edge] (3.800) -> (4.280);
        \end{tikzpicture}
    \end{example}
\end{frame}

\begin{frame}{Graphe}
    \begin{definition}
        Un \emph{graphe non orienté} est un ensemble de \emph{sommets} connectés par des \emph{arêtes}. Formellement, $G=(S,A)$ avec
        \begin{itemize}
            \item $S$ un ensemble de sommets 
            \item $A$ un ensemble de paires non-ordonnées de sommets
        \end{itemize}
    \end{definition}

    \begin{example}
        \begin{tikzpicture}
            \node[lettre] (1) at (0,0)  {1};
            \node[lettre] (2) at (0,-1) {2};
            \node[lettre] (3) at (1,-1) {3};
            \node[lettre] (4) at (1,0) {4};
            \node[lettre] (5) at (2,0) {5};
            \node[lettre] (6) at (2,-1) {6}; 
            \node (s) at (6,0) {6 sommets, 6 arêtes};
            \draw (1) -- (4);
            \draw (1) -- (2) -> (4);
            \draw (2) -- (3);
            \draw (6) -- (5);
            \draw (4) -- (3);
        \end{tikzpicture}
    \end{example}
\end{frame}

\begin{frame}{Vocabulaire}
    \begin{definition}
        Dans un graphe $G(S,A)$ orienté ou non, un sommet $y$ est dit \emph{adjacent} à $x$ si et seulement si $x,y \in A$
    \end{definition}
    \begin{example}
        \begin{tikzpicture}
            \node[lettre] (1) at (0,0)  {1};
            \node[lettre] (2) at (0,-1) {2};
            \node[lettre] (3) at (1,-1) {3};
            \node[lettre] (4) at (1,0) {4};
            \node[lettre] (5) at (2,0) {5};
            \node[lettre] (6) at (2,-1) {6}; 
            \draw[edge] (1) -> (4);
            \draw[edge] (1) -> (2) -> (4);
            \draw[edge] (2) -> (3);
            \draw[edge] (6) -> (5);
            \draw[edge] (4.260) -> (3.100);
            \draw[edge] (3.800) -> (4.280);
            \node (l) at (1,-2) {4 et 3 sont adjacents à 2};
        \end{tikzpicture}
        \begin{tikzpicture}
            \node[lettre] (1) at (0,0)  {1};
            \node[lettre] (2) at (0,-1) {2};
            \node[lettre] (3) at (1,-1) {3};
            \node[lettre] (4) at (1,0) {4};
            \node[lettre] (5) at (2,0) {5};
            \node[lettre] (6) at (2,-1) {6}; 
            \draw (1) -- (4);
            \draw (1) -- (2) -> (4);
            \draw (2) -- (3);
            \draw (6) -- (5);
            \draw (4) -- (3);
            \node (l) at (1,-2) {1, 4 et 3 sont adjacents à 2};
        \end{tikzpicture}

    \end{example}
\end{frame}

\begin{frame}{Vocabulaire}
    \begin{definition}
        Dans un graphe $G=(S,A)$ non orienté, le degré d'un sommet est le nombre de ses sommets adjacents
    \end{definition}
    \begin{example}
        \begin{tikzpicture}
            \node[lettre] (1) at (0,0)  {1};
            \node[lettre] (2) at (0,-1) {2};
            \node[lettre] (3) at (1,-1) {3};
            \node[lettre] (4) at (1,0) {4};
            \node[lettre] (5) at (2,0) {5};
            \node[lettre] (6) at (2,-1) {6}; 
            \draw (1) -- (4);
            \draw (1) -- (2) -> (4);
            \draw (2) -- (3);
            \draw (6) -- (5);
            \draw (4) -- (3);
            \node (l) at (6,0) {le sommet 3 est de degré 2};
            \node (ll) at (6,-1) {le sommet 3 est de degré 2};
        \end{tikzpicture}
    \end{example}
\end{frame}

\begin{frame}{Vocabulaire} 
    \begin{definition}
        Dans un graphe $G=(S,A)$ orienté, 
        \begin{itemize}
            \item le \emph{degré sortant} (ou extérieur) d'un sommet $x$, noté $\Gamma^+(x)$ est le nombre de ses sommets adjacents 
            \item le \emph{degré entrant} (ou intérieur) d'un sommet $x$, noté $\Gamma^-(x)$ est le nombre des sommets auxquels il est adjacent
        \end{itemize}
    \end{definition}
    \begin{example}
        \begin{tikzpicture}
            \node[lettre] (1) at (0,0)  {1};
            \node[lettre] (2) at (0,-1) {2};
            \node[lettre] (3) at (1,-1) {3};
            \node[lettre] (4) at (1,0) {4};
            \node[lettre] (5) at (2,0) {5};
            \node[lettre] (6) at (2,-1) {6}; 
            \draw[edge] (1) -> (4);
            \draw[edge] (1) -> (2) -> (4);
            \draw[edge] (2) -> (3);
            \draw[edge] (6) -> (5);
            \draw[edge] (4.260) -> (3.100);
            \draw[edge] (3.800) -> (4.280);
            \node (l) at (6,0) {le degré entrant de 2 est 1};
            \node (ll) at (6,-1) {le degré sortant de 2 est 2};
        \end{tikzpicture}        
    \end{example}
\end{frame}

\begin{frame}{Vocabulaire}
    \begin{definition}
        \label{def:chemin}
        Dans un graphe $G=(S,A)$ orienté ou non, un \emph{chemin} est une séquence de sommets $s_0,...s_n$ où chaque paire de sommets consécutifs $(s_k,s_{k+1})$ appartient à $A$. 
    \end{definition}
    \begin{example}
        \begin{tikzpicture}
            \node[lettre] (1) at (0,0)  {1};
            \node[lettre] (2) at (0,-1) {2};
            \node[lettre] (3) at (1,-1) {3};
            \node[lettre] (4) at (1,0) {4};
            \node[lettre] (5) at (2,0) {5};
            \node[lettre] (6) at (2,-1) {6}; 
            \draw[oeedge] (1) -> (4);
            \draw[edge] (1) -> (2) -> (4);
            \draw[edge] (2) -> (3);
            \draw[edge] (6) -> (5);
            \draw[oeedge] (4.260) -> (3.100);
            \draw[edge] (3.800) -> (4.280);
            \node (l) at (1,-2) {$(1,4,3)$ est un chemin};
        \end{tikzpicture}
        \begin{tikzpicture}
            \node[lettre] (1) at (0,0)  {1};
            \node[lettre] (2) at (0,-1) {2};
            \node[lettre] (3) at (1,-1) {3};
            \node[lettre] (4) at (1,0) {4};
            \node[lettre] (5) at (2,0) {5};
            \node[lettre] (6) at (2,-1) {6}; 
            \draw[eedge] (1) -- (4);
            \draw (1) -- (2);
            \draw (2) -- (4);
            \draw[eedge] (2) -- (3);
            \draw (6) -- (5);
            \draw[eedge] (4) -- (3);
            \node (l) at (1,-2) {$(1,2,3,4)$ est un chemin};
        \end{tikzpicture}
    \end{example}
\end{frame}

\begin{frame}{Vocabulaire}
    \begin{definition}
        Dans un graphe $G=(S,A)$ orienté ou non, un chemin \emph{simple} est un chemin sans répétition de sommets
    \end{definition}
    \begin{definition}
        Dans un graphe $G=(S,A)$ orienté ou non, un \emph{cycle} est un chemin $(s_0,...s_n)$ où $s_0 = s_n$ et $s_0,...s_{n-1}$ sont distincts
    \end{definition}
\end{frame}

% connexité



\begin{frame}{Vocabulaire}
    \begin{definition}
        \label{def:connexe}
        Un graphe $G=(S,A)$ \textbf{non orienté} est \emph{connexe} si et seulement si toute paire de sommets est reliée par au moins un chemin
    \end{definition}
    \begin{center}
        \includegraphics<1>[width=.49\textwidth]{fig/connexe.pdf}
        \includegraphics<2>[width=.49\textwidth]{fig/pasconnexe.pdf}
        \includegraphics<3>[width=.49\textwidth]{fig/pasconnexe2.pdf}
    \end{center}
    \visible<3>{Un graphe peut toujours être décomposé en une union disjointe de sous-graphes connexes : ses \emph{composantes connexes}.}
\end{frame}

\begin{frame}{Vocabulaire}
    \begin{definition}
        Un graphe $G=(S,A)$ \textbf{orienté} est \emph{fortement connexe} si et seulement si toute paire de sommets est reliée par au moins un chemin, c'est-à-dire si $\forall x,y \in S$ il existe un chemin de $x$ à $y$ \textbf{et} un chemin de $y$ à $x$
    \end{definition}
    \begin{block}{Propriété}
        La connectivité forte est une relation d'équivalence :

        \begin{itemize}
            \item elle est réflexive, symétrique et transitive 
            \item ses classes d'équivalences sont les composantes fortement connexes du graphe 
        \end{itemize}
    \end{block}
\end{frame}

\begin{frame}{Arbres}
    \begin{definition}
        \label{def:arbre}
        Un \emph{arbre} est un graphe non-orienté connexe et acyclique (ne contient pas de cycle)
    \end{definition}
    \begin{example}
        \begin{center}
        \begin{tikzpicture}
            \node[lettre] (1) at (0,0)  {1};
            \node[lettre] (2) at (0,-1) {2};
            \node[lettre] (3) at (1,-1) {3};
            \node[lettre] (4) at (1,0) {4};
            \node[lettre] (5) at (2,0) {5};
            \node[lettre] (6) at (2,-1) {6};
            \draw (1) -- (2);     
            \draw (2) -- (4);
            \draw (2) -- (3);
            \draw (3) -- (6);
            \draw (6) -- (5);     
        \end{tikzpicture}\hspace{1cm}
        \begin{tikzpicture}
            \node[lettre] (1) at (0,0)  {1};
            \node[lettre] (2) at (0,-1) {2};
            \node[lettre] (3) at (1,-1) {3};
            \node[lettre] (4) at (1,0) {4};
            \node[lettre] (5) at (2,0) {5};
            \node[lettre] (6) at (2,-1) {6};
            \draw (1) -- (2);     
            \draw (2) -- (4);
            \draw (2) -- (3);
            \draw (3) -- (6);
            \draw (6) -- (5);  
            \draw (3) -- (4);    
        \end{tikzpicture}\hspace{1cm}
        \begin{tikzpicture}
            \node[lettre] (1) at (0,0)  {1};
            \node[lettre] (2) at (0,-1) {2};
            \node[lettre] (3) at (1,-1) {3};
            \node[lettre] (4) at (1,0) {4};
            \node[lettre] (5) at (2,0) {5};
            \node[lettre] (6) at (2,-1) {6};
            \draw (1) -- (2);     
            \draw (2) -- (4);
            \draw (2) -- (3);
            \draw (6) -- (5);  
        \end{tikzpicture}
    \end{center}

    \end{example}
\end{frame}

\begin{frame}{Propriétés des arbres}
    \begin{itemize}
        \item Les arbres ont des propriétés fortes qui permettent de simplifier et accélérer certains algorithmes 
        \begin{block}{Propriété}
            Dans un arbre $(S,A)$, $|S| = |A| + 1$
        \end{block}
    \end{itemize}
    \begin{example}
        \begin{center}
            \begin{tikzpicture}
                \node[lettre] (1) at (0,0)  {1};
                \node[lettre] (2) at (0,-1) {2};
                \node[lettre] (3) at (1,-1) {3};
                \node[lettre] (4) at (1,0) {4};
                \node[lettre] (5) at (2,0) {5};
                \node[lettre] (6) at (2,-1) {6};
                \draw (1) -- (2);     
                \draw (2) -- (4);
                \draw (2) -- (3);
                \draw (3) -- (6);
                \draw (6) -- (5);     
            \end{tikzpicture}
        \end{center}            
    \end{example}
\end{frame}

\begin{frame}{Forêts}
    \begin{definition}
        \label{def:foret}
        Une \emph{forêt} est un graphe non-orienté acyclique
    \end{definition}
    \begin{example}
        \begin{center}
        \begin{tikzpicture}
            \node[lettre] (1) at (0,0)  {1};
            \node[lettre] (2) at (0,-1) {2};
            \node[lettre] (3) at (1,-1) {3};
            \node[lettre] (4) at (1,0) {4};
            \node[lettre] (5) at (2,0) {5};
            \node[lettre] (6) at (2,-1) {6};
            \draw (1) -- (2);     
            \draw (2) -- (4);
            \draw (2) -- (3);
            \draw (6) -- (5);  
        \end{tikzpicture}
    \end{center}
    \end{example}
    \pause 
    \begin{block}{Propriété}
        Les composantes connexes des forêts sont des arbres
    \end{block}

\end{frame}


\begin{frame}{Fermeture transitive}
    \begin{definition}
        Dans un graphe, orienté ou non, un sommet $y$ est dit \emph{accessible} depuis un sommmet $x$ si et seulement si
        il existe un chemin de $x$ à $y$
    \end{definition}

    \begin{block}{Propriété}
        On peut décider si $y$ est accessible depuis $x$ grâce à un parcours en profondeur depuis $x$ en temps linéaire
    \end{block}

    \begin{definition}
        La \emph{fermeture transitive} d'un graphe $G=(S,A)$ est un graphe $G^*=(S,A^*)$ avec les mêmes sommets mais tel qu'il existe un arc $(x,y) \in A^*$ si et seulement si il existe un chemin de $x$ à $y$ dans $G$
    \end{definition}
\end{frame}

\begin{frame}{Propriétés}
\begin{block}{Propriété}
    La fermeture transitive d'un graphe non orienté connexe (respectivement orienté fortement connexe) est le graphe complet
\end{block}

\begin{block}{Propriété}
    Si on connait $G^*$, la requête d'accessibilité se fait en temps constant
\end{block}
\end{frame}

% TODO remplacer les Adj[i] par $\Gamma(i)$ dans les algos 

\begin{frame}{Calcul de la fermeture transitive dans un graphe \emph{creux}}
\begin{columns}
\begin{column}{0.5\textwidth}
    \begin{algorithmic}
        \Function{visite}{l,i}
        \State $A^*[l,i]$ \gets True 
        \For{$j \in \Gamma(i)$} 
        \If{non $A^*[l,j]$}  
            \State visite(l,j)
        \EndIf
        \EndFor
        \EndFunction
    \end{algorithmic}
\end{column}
\begin{column}{0.5\textwidth}
    \begin{algorithmic}
        \Function{fermeture transitive}{}
            \State $A^*$ = [[False,False],...[False,False]]
            \For{$i \in S$} 
                \State visite(i,i)
            \EndFor \\
        \Return $A^*$
        \EndFunction
    \end{algorithmic}
\end{column}
\end{columns}
\end{frame}

\begin{frame}{Exemple}
\begin{center}
    \input{genfig/ft0}
\end{center}
\end{frame}

\begin{frame}{Exemple : fermeture réflexo-transitive}
\begin{center}
    \input{genfig/ft1}
\end{center}
    \end{frame}

\begin{frame}{Calcul de la fermeture transitive}
\begin{itemize}
    \item La calcul se fait en ${\cal O}(nm)$ avec $n$ parcours en profondeur
    \item En pratique, $A*$ est plutôt dense... On peut faire le calcul de façon matricielle
    \item cf. algorithme de Floyd-Warshall en TD 
\end{itemize}
\end{frame}


% TODO placé là mais à déplacer après le merge 
% calcul des composantes fortement connexes 

\begin{frame}{Calcul des composantes fortements connexes : un peu d'histoire}

\begin{itemize}
    \item Années 60 : un problème classique 
    \begin{itemize}
        \item mais sans solution efficace (polynomiale)
        \item la complexité du problème n'est pas connue 
    \end{itemize}
    \item 1972 : algorithme linéaire proposé par Trojan 
    \begin{itemize}
        \item simple modification du parcours en profondeur 
        \item difficile à comprendre 
    \end{itemize}
    \item années 80 : algorithme linéaire de Kosajaru 
    \begin{itemize}
        \item plus simple à comprendre (2 parcours)
        \item la légende dit qu'il avait oublié ses notes de cours et qu'il a inventé l'algorithme en préparant son cours 
    \end{itemize}
\end{itemize}
\end{frame}

\begin{frame}{Notion de graphe inverse}
\begin{definition}
    Le graphe \emph{inverse} d'un graphe $G=(S,A)$ est le graphe $G'=(S,A')$ possédant les mêmes sommets mais possédant un arc $(i,j) \in A'$ si et seulement si $(j,i) \in A$    
\end{definition}

\begin{example}
    \begin{tikzpicture}
        \node[lettre] (1) at (0,0)  {1};
        \node[lettre] (2) at (0,-1) {2};
        \node[lettre] (3) at (1,-1) {3};
        \node[lettre] (4) at (1,0) {4};
        \node[lettre] (5) at (2,0) {5};
        \node[lettre] (6) at (2,-1) {6}; 
        \draw[edge] (1) -> (4);
        \draw[edge] (1) -> (2) -> (4);
        \draw[edge] (2) -> (3);
        \draw[edge] (6) -> (5);
        \draw[edge] (4.260) -> (3.100);
        \draw[edge] (3.800) -> (4.280);
        \node (l) at (4.5,0) { et son graphe inverse};
    \end{tikzpicture}        
    \begin{tikzpicture}
        \node[lettre] (1) at (0,0)  {1};
        \node[lettre] (2) at (0,-1) {2};
        \node[lettre] (3) at (1,-1) {3};
        \node[lettre] (4) at (1,0) {4};
        \node[lettre] (5) at (2,0) {5};
        \node[lettre] (6) at (2,-1) {6}; 
        \draw[edge] (4) -> (1);
        \draw[edge] (4) -> (2) -> (1);
        \draw[edge] (3) -> (2);
        \draw[edge] (5) -> (6);
        \draw[edge] (4.260) -> (3.100);
        \draw[edge] (3.800) -> (4.280);
    \end{tikzpicture}        
\end{example}

\begin{block}{Propriété}
    Les composantes fortement connexes d'un graphe sont identiques à celles de son graphe inverse
\end{block}

\end{frame}

\begin{frame}{Algorithme de Kosaraju : principe}

\begin{enumerate}
    \item Parcours en profondeur de $G$ pour retenir un ordre suffixe
    \begin{itemize}
        \item i.e. quand on visite $b$ depuis $a$, on note $b$, $a$
    \end{itemize}
    \item Calcul de $G'$ l'inverse de $G$
    \item Parcours en profondeur sur $G'$ mais avec modification de la boucle principale : on prend la boucle principale par l'inverse de l'ordre suffixe établi à la première étape
    \item Les composantes fortement connexes sont les arbres de la forêt du second parcours
\end{enumerate}

\end{frame}

\begin{frame}[fragile]
    \frametitle{Algorithme de Kosaraju (partie 1)}
        \begin{columns}
            \begin{column}{.5\textwidth}
                \begin{algorithmic}[1]
                    \Function{visite}{$i$ : sommet}
                    \State vu[i] \gets true
                    \For{$j \in Adj[i]$}
                        \If{$vu[j] = false$}
                            \State visite(j)
                        \EndIf
                    \EndFor
                    \State \textcolor{blue}{os.add(i)}
                    \EndFunction
                \end{algorithmic}
            \end{column}
            \begin{column}{.5\textwidth}
                \begin{algorithmic}[1]
                    \State vu \gets [$false$,...,$false$]
                    \State \textcolor{blue}{os \gets []}
                    \For{$i \in S$}
                        \If{$vu[i] = false$}
                        \State visite(i)
                        \EndIf
                    \EndFor
                    \end{algorithmic}            
            \end{column}
        \end{columns}    
    \end{frame}

    \begin{frame}[fragile]
        \frametitle{Algorithme de Kosaraju (partie 2)}
            \begin{columns}
                \begin{column}{.5\textwidth}
                    \begin{algorithmic}[1]
                        \Function{visite\_k}{ginv,$i$ : sommet,ncb}
                        \State vu[i] \gets true
                        \State cfc[i] \gets ncb
                        \For{$j \in Adj[i]$}
                            \If{$vu[j] = false$}
                                \State visite\_k(ginv,j,ncb)
                            \EndIf
                        \EndFor
                        \EndFunction
                    \end{algorithmic}
                \end{column}
                \begin{column}{.5\textwidth}
                    \begin{algorithmic}[1]
                        \While{!os.empty()}
                        \State i \gets os.pop() 
                        \If{vu[i] == False}
                            \State cfc[i] \gets ncb
                            \State visite\_k(ginv,i,ncb)
                            \State ncb \gets ncb+1
                        \EndIf
                        \EndWhile
                        \end{algorithmic}            
                \end{column}
            \end{columns}    
        \end{frame}
    
% TODO faut-il inclure l'algorithme de Tarjan à titre de comparaison ?

\begin{frame}[fragile]
\frametitle{Algorithme de Kosaraju}
\begin{pythoncode}
# Parcours en profondeur avec stockage de l'ordre suffixe 
def visite(g,i):
    global ordre_suffixe
    i["vu"] = True
    for j in g.successors(i):
        if g.vs[j]["vu"] == False:
            visite(g,g.vs[j])
    ordre_suffixe.append(i["nom"])


\end{pythoncode}
\end{frame}


\begin{frame}[fragile]
\frametitle{Algorithme de Kosaraju}
\begin{pythoncode}    
# initialisation
for s in g1.vs:
    s["vu"] = False
ordre_suffixe = []
for i in g1.vs:
    if i["vu"] == False:
        visite(g1,i)

\end{pythoncode}
\end{frame}

\begin{frame}[fragile]
\frametitle{Algorithme de Kosaraju}
\begin{pythoncode}    

#inversion du graphe 
ginv = ig.Graph(24,[],True)
n = 1
for v in ginv.vs:
    v["nom"] = str(n) 
    v["vu"] = False 
    n = n + 1
for e in g1.es: 
    ginv.add_edge(e.target,e.source)

ncb = 1

while ordre_suffixe:
    i = ordre_suffixe.pop()
    e = ginv.vs.select(nom = i)[0]
    if (e["vu"] == False):
        e["comp"] = ncb
        visite_k(ginv,e,ncb)
        ncb = ncb+1
\end{pythoncode}
\end{frame}

\begin{frame}[fragile]
    \frametitle{Algorithme de Kosaraju}
    \begin{pythoncode}   
    
    def visite_k(ginv, v, ncb):
        v["vu"] = True
        v["comp"] = ncb
        for i in v.successors():
            if i["vu"] == False:
                visite_k(ginv, i, ncb)
    \end{pythoncode}
    \end{frame}
    

