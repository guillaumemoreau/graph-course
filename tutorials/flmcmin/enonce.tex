%%%%%%%%%%%%%%%%%%%%%%%%%%%%%%%%%%%%%%%%%
% Short Sectioned Assignment
% LaTeX Template
% Version 1.0 (5/5/12)
%
% This template has been downloaded from:
% http://www.LaTeXTemplates.com
%
% Original author:
% Frits Wenneker (http://www.howtotex.com)
%
% License:
% CC BY-NC-SA 3.0 (http://creativecommons.org/licenses/by-nc-sa/3.0/)
%
%%%%%%%%%%%%%%%%%%%%%%%%%%%%%%%%%%%%%%%%%

%----------------------------------------------------------------------------------------
%	PACKAGES AND OTHER DOCUMENT CONFIGURATIONS
%----------------------------------------------------------------------------------------

\documentclass[paper=a4, fontsize=11pt]{scrartcl} % A4 paper and 11pt font size

\usepackage{fourier} % Use the Adobe Utopia font for the document - comment this line to return to the LaTeX default
\usepackage[french]{babel} % English language/hyphenation
\usepackage{amsmath,amsfonts,amsthm} % Math packages

\usepackage{lipsum} % Used for inserting dummy 'Lorem ipsum' text into the template
\usepackage{graphicx}
\usepackage{sectsty} % Allows customizing section commands
\allsectionsfont{\centering \normalfont\scshape} % Make all sections centered, the default font and small caps

\usepackage{fancyhdr} % Custom headers and footers
\pagestyle{fancyplain} % Makes all pages in the document conform to the custom headers and footers
\fancyhead{} % No page header - if you want one, create it in the same way as the footers below
\fancyfoot[L]{} % Empty left footer
\fancyfoot[C]{} % Empty center footer
\fancyfoot[R]{\thepage} % Page numbering for right footer
\renewcommand{\headrulewidth}{0pt} % Remove header underlines
\renewcommand{\footrulewidth}{0pt} % Remove footer underlines
\setlength{\headheight}{13.6pt} % Customize the height of the header

\numberwithin{equation}{section} % Number equations within sections (i.e. 1.1, 1.2, 2.1, 2.2 instead of 1, 2, 3, 4)
\numberwithin{figure}{section} % Number figures within sections (i.e. 1.1, 1.2, 2.1, 2.2 instead of 1, 2, 3, 4)
\numberwithin{table}{section} % Number tables within sections (i.e. 1.1, 1.2, 2.1, 2.2 instead of 1, 2, 3, 4)

\setlength\parindent{0pt} % Removes all indentation from paragraphs - comment this line for an assignment with lots of text

%----------------------------------------------------------------------------------------
%	TITLE SECTION
%----------------------------------------------------------------------------------------

\newcommand{\horrule}[1]{\rule{\linewidth}{#1}} % Create horizontal rule command with 1 argument of height

\title{	
\normalfont \normalsize 
\textsc{Ecole Centrale de Nantes} \\ [25pt] % Your university, school and/or department name(s)
\horrule{0.5pt} \\[0.4cm] % Thin top horizontal rule
\huge Option Informatique - TD MADIS \\ % The assignment title
\horrule{2pt} \\[0.5cm] % Thick bottom horizontal rule
}

%\author{Durée 1h - Aucun document autorisé} % Your name

%\date{\normalsize\today} % Today's date or a custom date

\begin{document}

\maketitle % Print the title

\section{Grossiste en fleurs}

Un grossiste en fleurs assure le transport depuis Marseille et Valence vers Paris. Ce transport est effectué par camionnette de Valence à Carpentras et de Marseille à Carpentras ou Grenoble, puis par train ou avion de Carpentras à Paris et par train de Grenoble à Paris. On désire étudier le transport global par semaine de façon à envoyer un maximum de fleurs à Paris à coût minimal bien entendu.

Les camionnettes disponibles à Valence permettent d'acheminer au plus 400 cartons par semaine vers Carpentras au coût unitaire de 2 ; celles de Marseille 400 cartons dont la moitié vers Carpentras au coût unitaire de 3 et l'autre moitié vers Grenoble au coût de 2. De plus, on doit limiter le transport par le train à 300 cartons dont 200 au plus depuis Carpentras, le coût unitaire étant alors de 10 depuis Grenoble et de 9 depuis Carpentras. Par avion, le coût unitaire est de 16. 

La production de Marseille ne peut dépasser 200 cartons, celle de Valence est toujours excédentaire.

\begin{enumerate}
\item Modéliser le problème sous forme de graphe 
\item Après avoir simplifié au maximum le graphe, résoudre le problème à l'aide de l'algorithme de Roy.

\end{enumerate}

\end{document}
%%% Local Variables: 
%%% mode: latex
%%% TeX-master: t
%%% End: 
