% représentation des graphes 

\section{Représentation des graphes}

\begin{frame}{Représentation des graphes}
    \begin{itemize}
        \item On s'intéresse à la représentation des graphes orientés ou non
        \begin{itemize}
            \item revient au même à une redondance près
        \end{itemize}
        \item On suppose $S={1,...,n}$
        \begin{itemize}
            \item au besoin on utilise une structure de dictionnaire (map) pour faire un lien entre sommets et indexes
        \end{itemize}
    \end{itemize}
\end{frame}

\begin{frame}{Opérations de base}
    \begin{itemize}
        \item Créer un graphe à $n$ sommets, sans arcs
        \item Ajouter un arc
        \item Tester l'existence d'un arc 
        \item Parcourir les sommets adjacents à un sommet 
        \item Parcourir \textbf{tous} les sommets en explorant le graphe
        \begin{itemize}
            \item Algorithme de \emph{parcours de graphe}
        \end{itemize}
    \end{itemize}
\end{frame}

\begin{frame}{Représentation par matrice d'adjacences}
    \begin{itemize}
        \item une matrice $n \times n$ à valeurs dans $\{0,1\}$
    \end{itemize}
    \begin{equation*}
        M_{i,j} = \left\{ 
            \begin{array}{ll}
                1 & \mbox{si } (i,j) \in A \\
                0 & \mbox{sinon} \\
            \end{array}
            \right.
        \end{equation*}
\begin{itemize}
    \item Bien adapté aux graphes denses 
\end{itemize}
\end{frame}

\begin{frame}{Exemple}
    \begin{columns}
        \begin{column}{.5\textwidth}
            \begin{tikzpicture}
                \node[lettre] (1) at (1,0)  {1};
                \node[lettre] (2) at (3,-0) {2};
                \node[lettre] (3) at (3,-1) {3};
                \node[lettre] (4) at (2,-1) {4};
                \node[lettre] (5) at (0,-1) {5};
                \node[lettre] (6) at (1,-2) {6}; 
                \draw[edge] (1) -> (2);
                \draw[edge] (1) -> (4);
                \draw[edge] (1) -> (5);
                \draw[edge] (5) -> (6);
                \draw[edge] (6) -> (4);
                \draw[edge] (3) -> (4);
                \draw[edge] (2.260) -> (3.100);
                \draw[edge] (3.800) -> (2.280);
                \draw[edge] (5.10) -> (4.170);
                \draw[edge] (4.190) -> (5.350);
            \end{tikzpicture}
        \end{column}
        \begin{column}{.5\textwidth}
            \begin{equation*}
                m = \left(
                  \begin{array}{cccccc}
                    0 & 1 & 0 & 1 & 1 & 0 \\
                    0 & 0 & 1 & 0 & 0 & 0 \\
                    0 & 1 & 0 & 1 & 0 & 0 \\
                    0 & 0 & 0 & 0 & 1 & 0 \\
                    0 & 0 & 0 & 1 & 0 & 1 \\
                    0 & 0 & 0 & 1 & 0 & 0 \\
                  \end{array}
                  \right)
              \end{equation*}
        \end{column}
    \end{columns}    
\end{frame}

\begin{frame}{Exemple}
    \begin{columns}
        \begin{column}{.5\textwidth}
            \begin{tikzpicture}
                \node[lettre] (1) at (1,0)  {1};
                \node[lettre] (2) at (3,-0) {2};
                \node[lettre] (3) at (3,-1) {3};
                \node[lettre] (4) at (2,-1) {4};
                \node[lettre] (5) at (0,-1) {5};
                \node[lettre] (6) at (1,-2) {6}; 
                \draw[edge] (1) -> (2);
                \draw[edge] (1) -> (4);
                \draw[edge] (1) -> (5);
                \draw[edge] (5) -> (6);
                \draw[oeedge] (6) -> (4);
                \draw[edge] (3) -> (4);
                \draw[edge] (2.260) -> (3.100);
                \draw[edge] (3.800) -> (2.280);
                \draw[edge] (5.10) -> (4.170);
                \draw[edge] (4.190) -> (5.350);
            \end{tikzpicture}
        \end{column}
        \begin{column}{.5\textwidth}
            \begin{equation*}
                m = \left(
                  \begin{array}{cccccc}
                    0 & 1 & 0 & 1 & 1 & 0 \\
                    0 & 0 & 1 & 0 & 0 & 0 \\
                    0 & 1 & 0 & 1 & 0 & 0 \\
                    0 & 0 & 0 & 0 & 1 & 0 \\
                    0 & 0 & 0 & 1 & 0 & 1 \\
                    0 & 0 & 0 & \textcolor{blue}{\textbf{1}} & 0 & 0 \\
                  \end{array}
                  \right)
              \end{equation*}
        \end{column}
    \end{columns}    
\end{frame}

\begin{frame}{Quelques chiffres}
    \begin{itemize}
        \item $G=(S,A)$ à $n$ sommets et $m$ arcs 
        \item Dans un graphe orienté : $m \leq n^2$ 
        \begin{itemize}
            \item Si $m=n^2$, le graphe est \emph{complet}
        \end{itemize}
        \item Dans un graphe non-orienté : $m \leq \frac{n(n+1)}{2}$
        \begin{itemize}
            \item Graphe complet si égalité 
        \end{itemize}
        \item Si $m ~ n^2$, on parle de graphe \emph{dense}
    \end{itemize}
\end{frame}

\begin{frame}{Représentation par liste d'adjacences}
    \begin{itemize}
        \item Un tableau $Adj$ de $|S|$ listes de sommets tel que pour tout sommet $i$, $Adj[i]$ contient les adjacents de $i$
        \begin{equation*}
            Adj[i] = \{  j | (i,j) \in A \}
        \end{equation*}
        \item Bien adapté aux graphes peu denses (\textit{sparse})
    \end{itemize}
\end{frame}

\begin{frame}{Exemple}
    \begin{columns}
        \begin{column}{.5\textwidth}
            \begin{tikzpicture}
                \node[lettre] (1) at (1,0)  {1};
                \node[lettre] (2) at (3,-0) {2};
                \node[lettre] (3) at (3,-1) {3};
                \node[lettre] (4) at (2,-1) {4};
                \node[lettre] (5) at (0,-1) {5};
                \node[lettre] (6) at (1,-2) {6}; 
                \draw[edge] (1) -> (2);
                \draw[edge] (1) -> (4);
                \draw[edge] (1) -> (5);
                \draw[edge] (5) -> (6);
                \draw[edge] (6) -> (4);
                \draw[edge] (3) -> (4);
                \draw[edge] (2.260) -> (3.100);
                \draw[edge] (3.800) -> (2.280);
                \draw[edge] (5.10) -> (4.170);
                \draw[edge] (4.190) -> (5.350);
            \end{tikzpicture}
        \end{column}
        \begin{column}{.5\textwidth}
            \begin{equation*}
                  \begin{array}{ccl}
                    Adj[1] & = & \{2, 4, 5 \} \\
                    Adj[2] & = & \{3 \} \\
                    Adj[3] & = & \{ 2, 4 \} \\
                    Adj[4] & = & \{ 5 \} \\
                    Adj[5] & = & \{ 4, 6 \} \\
                    Adj[6] & = & \{ 4 \} \\
                  \end{array}
              \end{equation*}
            Note : les listes sont ordonnées ici, cela n'a rien d'obligatoire
        \end{column}
    \end{columns}    
\end{frame}

\begin{frame}{Autres représentations}
    \begin{itemize}
        \item une seule liste de tous les arcs/arêtes 
        \begin{equation*}
            TA = \left[
              \begin{array}{cccccccccc}
              1 & 1 & 1 & 2 & 3 & 3 & 4 & 5 & 5 & 6 \\
              2 & 4 & 5 & 3 & 2 & 4 & 5 & 4 & 6 & 4 \\
              \end{array}
              \right]
          \end{equation*}          
        \item toute forme de compression possible 
        \begin{itemize}
            \item \textit{façon RLE}
            \begin{equation*}
                T_1 = \left[
                  \begin{array}{cccccccccc}
                  2 & 4 & 5 & 3 & 2 & 4 & 5 & 4 & 6 & 4 \\
                  \end{array}
                  \right]
              \end{equation*}
              
              \begin{equation*}
                T_2 = \left[
                  \begin{array}{cccccc}
                   1 & 4 & 5 & 7 & 8 & 10 \\
                  \end{array}
                  \right]
              \end{equation*}
              
        \end{itemize}
    \end{itemize}
    \pause 
    \alert{Dans tous les cas, les coûts de manipulation sont différents ! Il va falloir faire un compromis entre coût mémoire / coût en temps de calcul / coût en développement des algorithmes}
\end{frame}

% aborder l'efficacité des représentations 

\begin{frame}{Efficacité des représentations}
    \begin{center}
        \begin{tabular}{|l|c|c|c|}
            \hline
            Représentation & Espace & $(i,j)\in A$ ? & $\Gamma^+(a)$ \\
            \hline
            liste d'arcs & ${\cal O}(m)$ & ${\cal O}(m)$ & ${\cal O}(m)$ \\
            \hline
            matrice d'adjacences & ${\cal O}(n^2)$ & ${\cal O}(1)$ & ${\cal O}(n)$  \\
            \hline  
            tableau d'adjacences & ${\cal O}(n+m)$ & ${\cal O}(\Gamma(i))$ &  ${\cal O}(\Gamma(i))$ \\ 
            \hline
        \end{tabular}
    \end{center}

    \begin{itemize}
        \item Questions
        \begin{itemize}
            \item Que se passe-t-il si on cherche $\Gamma^-(i)$ ?
            \item Que se passe-t-il si la liste d'arcs est triée ?
        \end{itemize}
    \end{itemize}
\end{frame}

\begin{frame}{Efficacité des représentations}
    \begin{center}
        \begin{tabular}{|l|c|c|c|}
            \hline
            Représentation & Espace & $(i,j)\in A$ ? & $\Gamma^+(a)$ \\
            \hline
            liste d'arcs & ${\cal O}(m)$ & ${\cal O}(m)$ & ${\cal O}(m)$ \\
            \hline 
            \textcolor{blue}{liste triée} & ${\cal O}(m)$ & ${\cal O}(\log n + \log \Gamma^+(i))$ & ${\cal O}(\log n + \Gamma^+(i))$ \\
            \hline
            $M_{i,j}$ & ${\cal O}(n^2)$ & ${\cal O}(1)$ & ${\cal O}(n)$  \\
            \hline  
            $Adj[i]$ & ${\cal O}(n+m)$ & ${\cal O}(\Gamma(i))$ &  ${\cal O}(\Gamma(i))$ \\ 
            \hline
        \end{tabular}
    \end{center}

    \begin{itemize}
        \item En pratique, les graphes sont rarement denses 
    \end{itemize}
\end{frame}


