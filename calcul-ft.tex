\begin{frame}{Calcul de la fermeture transitive dans un graphe \emph{creux}}
    \begin{columns}
    \begin{column}{0.5\textwidth}
        \begin{algorithmic}
            \Function{visite}{l,i}
            \State $A^*[l,i]$ \gets True 
            \For{$j \in \Gamma(i)$} 
            \If{non $A^*[l,j]$}  
                \State visite(l,j)
            \EndIf
            \EndFor
            \EndFunction
        \end{algorithmic}
    \end{column}
    \begin{column}{0.5\textwidth}
        \begin{algorithmic}
            \Function{fermeture transitive}{}
                \State $A^*$ = [[False,False],...[False,False]]
                \For{$i \in S$} 
                    \State visite(i,i)
                \EndFor \\
            \Return $A^*$
            \EndFunction
        \end{algorithmic}
    \end{column}
    \end{columns}
    \end{frame}
    
    \begin{frame}{Exemple}
    \begin{center}
        \input{genfig/ft0}
    \end{center}
    \end{frame}
    
    \begin{frame}{Exemple : fermeture réflexo-transitive}
    \begin{center}
        \input{genfig/ft1}
    \end{center}
        \end{frame}
    
    \begin{frame}{Calcul de la fermeture transitive}
    \begin{itemize}
        \item La calcul se fait en ${\cal O}(nm)$ avec $n$ parcours en profondeur
        \item En pratique, $A*$ est plutôt dense... On peut faire le calcul de façon matricielle
        \item cf. algorithme de Floyd-Warshall en TD 
    \end{itemize}
    \end{frame}
    