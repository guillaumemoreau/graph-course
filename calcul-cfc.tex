% calcul des composantes fortement connexes 

\begin{frame}{Calcul des composantes fortements connexes : un peu d'histoire}

\begin{itemize}
    \item Années 60 : un problème classique 
    \begin{itemize}
        \item mais sans solution efficace (polynomiale)
        \item la complexité du problème n'est pas connue 
    \end{itemize}
    \item 1972 : algorithme linéaire proposé par Trojan 
    \begin{itemize}
        \item simple modification du parcours en profondeur 
        \item difficile à comprendre 
    \end{itemize}
    \item années 80 : algorithme linéaire de Kosajaru 
    \begin{itemize}
        \item plus simple à comprendre (2 parcours)
        \item la légende dit qu'il avait oublié ses notes de cours et qu'il a inventé l'algorithme en préparant son cours 
    \end{itemize}
\end{itemize}
\end{frame}

\begin{frame}{Notion de graphe inverse}
\begin{definition}
    Le graphe \emph{inverse} d'un graphe $G=(S,A)$ est le graphe $G'=(S,A')$ possédant les mêmes sommets mais possédant un arc $(i,j) \in A'$ si et seulement si $(j,i) \in A$    
\end{definition}

\begin{example}
    \begin{tikzpicture}
        \node[lettre] (1) at (0,0)  {1};
        \node[lettre] (2) at (0,-1) {2};
        \node[lettre] (3) at (1,-1) {3};
        \node[lettre] (4) at (1,0) {4};
        \node[lettre] (5) at (2,0) {5};
        \node[lettre] (6) at (2,-1) {6}; 
        \draw[edge] (1) -> (4);
        \draw[edge] (1) -> (2) -> (4);
        \draw[edge] (2) -> (3);
        \draw[edge] (6) -> (5);
        \draw[edge] (4.260) -> (3.100);
        \draw[edge] (3.800) -> (4.280);
        \node (l) at (4.5,0) { et son graphe inverse};
    \end{tikzpicture}        
    \begin{tikzpicture}
        \node[lettre] (1) at (0,0)  {1};
        \node[lettre] (2) at (0,-1) {2};
        \node[lettre] (3) at (1,-1) {3};
        \node[lettre] (4) at (1,0) {4};
        \node[lettre] (5) at (2,0) {5};
        \node[lettre] (6) at (2,-1) {6}; 
        \draw[edge] (4) -> (1);
        \draw[edge] (4) -> (2) -> (1);
        \draw[edge] (3) -> (2);
        \draw[edge] (5) -> (6);
        \draw[edge] (4.260) -> (3.100);
        \draw[edge] (3.800) -> (4.280);
    \end{tikzpicture}        
\end{example}

\begin{block}{Propriété}
    Les composantes fortement connexes d'un graphe sont identiques à celles de son graphe inverse
\end{block}

\end{frame}

\begin{frame}{Algorithme de Kosaraju : principe}

\begin{enumerate}
    \item Parcours en profondeur de $G$ pour retenir un ordre suffixe
    \begin{itemize}
        \item i.e. quand on visite $b$ depuis $a$, on note $b$, $a$
    \end{itemize}
    \item Calcul de $G'$ l'inverse de $G$
    \item Parcours en profondeur sur $G'$ mais avec modification de la boucle principale : on prend la boucle principale par l'inverse de l'ordre suffixe établi à la première étape
    \item Les composantes fortement connexes sont les arbres de la forêt du second parcours
\end{enumerate}

\end{frame}

% TODO faut-il inclure l'algorithme de Tarjan à titre de comparaison ?

